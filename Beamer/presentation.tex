\documentclass{beamer}
\usepackage[T1]{fontenc}
\usepackage[utf8]{inputenc}
\usepackage[italian]{babel}
\usepackage{tikz-cd,wrapfig}
\usepackage{tcolorbox}

\usepackage[style=numeric, maxnames=4,backend=bibtex]{biblatex}
% other styles: numeric authortitle
\addbibresource{../biblio.bib}

\title{Greither unit undex}
\author{Giacomo Borin}
%\subtitle{}
%\date{...}
\institute{Università di Trento}
%\titlegraphic{...} 

\usetheme{Madrid}
\usecolortheme{rose}
\setbeamercovered{dynamic}

\AtBeginSection{
	\begin{frame} 
		\sectionpage
	\end{frame}
}


% Theorem definitons 
\theoremstyle{plain}
\newtheorem{teo}{Teorema}[section]
\newtheorem{lem}[teo]{Lemma}
\newtheorem{prop}[teo]{Proposizione}
\newtheorem{cor}[teo]{Corollario}
\newtheorem*{form}{Formula}

\theoremstyle{remark}
\newtheorem{rem}{Osservazione}
\newtheorem{rems}[rem]{Remarks}

\theoremstyle{definition}
\newtheorem{deff}[teo]{Definizione}
\newtheorem{idea}{Idea}
\newtheorem*{nota}{Notazione}

% Commands 
\newcommand{\noqed}{\let\qed\relax}
\newcommand{\PS}{\mathcal{P}_S}
\newcommand{\Z}{\mathbb{Z}}
\newcommand{\F}{\mathbb{F}}
\newcommand{\K}{\mathbb{K}}
\newcommand{\ZZ}[1]{\mathbb{Z}_{#1}}
\newcommand{\Q}{\mathbb{Q}}
\newcommand{\C}{\mathbb{C}}
\newcommand{\R}{\mathbb{R}}

\DeclareMathOperator*{\eqb }{=}
\DeclareMathOperator{\Gal}{Gal}
\DeclareMathOperator{\ord}{ord}


\begin{document}

	\begin{frame}[plain]
	    \maketitle
	\end{frame}
	
	\begin{frame}{Introduzione}
		In questo lavoro ho rielaborato l'articolo: 
		\pause
%		\printbibliograph
		\begin{exampleblock}{}
			\fullcite{GRE}
		\end{exampleblock}	
		\pause
		Completando i prerequisiti richiesti per la comprensione e implementando alcuni calcoli in \href{https://www.sagemath.org}{\includegraphics[height= 11pt]{../images/sage.png} }
		
	\end{frame}
	\section{Prereqisiti}

	\begin{frame}{Il gruppo delle unità}
		\begin{block}{Primo oggetto di interesse: $ E_K $}
			Il gruppo delle unità di $ K $, il sottocampo reale massimo di $ \Q (\zeta_n )  $
		\end{block}
		\pause
		\begin{itemize}
			\item $ \zeta_n $  l'$ n $-esima radice ciclotomica (con $ n \not \equiv 2 \mod 4 $) \pause
			\item $ K = \Q (\zeta_n + \zeta_n^{-1})$ \pause
			\item $ O_K = \Z[\zeta_n + \zeta_n^{-1}] $ \pause
			\item $ E_K = O _K ^\ast$ , cioè l'insieme degli elementi invertibili 
		\end{itemize}
	\end{frame}
	
	\begin{frame}
		\begin{proof} \noqed
			\begin{itemize}
			\item $  \zeta_n + \zeta_n^{-1}  $ è \textbf{reale}:
			\[ \overline{\zeta_n + \zeta_n^{-1}} = \overline{\zeta_n} + \overline{\zeta_n}^{-1} = \zeta_n^{-1} + \zeta_n \] \pause
			\item L'indice $ [\Q(\zeta_n) : K] $ vale \textbf{2}, ed è quindi minimale. \pause 
			Il suo polinomio minimo è: 
			\[ f(x) = (x-\zeta)(x- \zeta^{-1}) = x^2 - (\zeta + \zeta ^{-1})x +1   \]
			\end{itemize}
		\end{proof}
	\end{frame}
	
	%
	\begin{frame}[fragile]
		\begin{proof}
			\begin{itemize}
				\item Un sottocampo con queste caratteristiche è \textbf{unico}:
				\pause
			\end{itemize}
			\begin{center}
				\begin{tikzcd}[column sep=small]
					& {\Q (\zeta)}                            &                               \\
					{K} \arrow[ru, "2"] \arrow[r] & {H} \arrow[u, "2" ] & {K'} \arrow[lu, "2"'] \arrow[l] \\
					& {\Q} \arrow[lu] \arrow[ru]      &                              
				\end{tikzcd}
			\end{center}
		\end{proof}
		
	\end{frame}
	
	
	\begin{frame}
		\begin{prop}
			Il gruppo di Galois di $ K $ è isomorfo a $ \Z_n^\ast / \{\pm 1\} $
		\end{prop}
		\pause 
		D'ora in poi indicheremo $ G_0 := \Gal( \Q(\zeta_n) / \Q) $ e $ G := \Gal( K / \Q) $
		
	\end{frame}
	
	\begin{frame}{Il numero delle classi}
		\begin{deff}
				Se $ \K $ è un campo numerico possiamo definire l'\textbf{ideal class group} come il quoziente $ \mathcal{F}_\K / \mathcal{P}_\K $ dove:
			\begin{itemize}
				\item[$ \mathcal{F}_\K $] è il gruppo degli ideali frazionari non nulli di $ O_\K $, 
				\item[$ \mathcal{P}_\K $] è il gruppo degli ideali principali
			\end{itemize}
		\end{deff} 
		\pause 
		Si può mostrare che questo gruppo è finito e definiamo il \textbf{numero delle classi} come:
			\[ h_K = |  \mathcal{F}_\K / \mathcal{P}_\K | \]
	\end{frame}
	
	\begin{frame}{Unità circolari}
		Se considero il gruppo generato da $ \{ -1 ,  \zeta , \, 1 - \zeta ^a \text{ for } a = 1, ... , n-1 \} $ e lo interseco con $ E_K $ ottengo il gruppo delle \textbf{unità circolari}\pause
		\begin{exampleblock}{}
			Sinnot ha mostrato che esiste $ a \in \Z $ tale che:
			\[ [E_K : C_K] = 2^a k_K\]
			\nocite{SIN}
		\end{exampleblock}
	\end{frame}
	
	\begin{frame}{Obbiettivo dell'articolo}
		\begin{alertblock}{}
			 	Costruire \textbf{esplicitamente} un gruppo  $ C' $ con indice $ [ E_K : C' ] $ finito che sia \textbf{ottimale}
		\end{alertblock}
	\end{frame}
	
	\subsection{Caratteri di Dirichlet}
	
	\begin{frame}{Caratteri di Dirichlet}
		\begin{deff}
			 	Dato un gruppo \textit{X} e un campo $ \F $ un carattere di Dirchlet è un omomorfismo di gruppi $ \chi :X \to \F ^*  $
		 \end{deff}
		 \pause
		 Possiamo anche usare l'isomorfismo $ G_0 \simeq \Z_n ^ *$ e definire $\chi$ come omomorfismo di anelli da $ \Z_n $ in $ \C $ (con $\chi$ nulla sugli elementi invertibili)
		 \pause
		 \begin{exampleblock}{}
		 Il 'periodo' di un caratte è detto conduttore (in inglese \textbf{conductor}) e si indica con $ f_\chi $
		 \end{exampleblock}
	\end{frame}
	
	\begin{frame}{Anelli gruppali}
		\begin{deff}
			Dati un gruppo moltiplicativo $ X $ e un anello $ R $ possiamo definire l'\textbf{anello gruppale} $ R[X] $ come l'$ R $-modulo libero con base $ X $, sul quale definiamo un operazione di moltiplicazione inducendola da quella di $ X $
		\end{deff}
		\pause
		Nel nostro caso useremo $ \Z [G_0] $ e $ \Z [G] $, sui quali possiamo sempre esterndere il carattere $\chi$ (perchè definito sulla base)
	\end{frame}
	
	\begin{frame}{Anelli gruppali}
		\begin{nota}
				Dati $ z \in \Q (\zeta) $ e $ f \in \Z[G_0]  $ è ben definita la nutazione esponenziale $ x^f $, \pause infatti dati $ g\in G_0 $ abbiamo una buona definizione per  $ z^g = g(z) $ e $ z ^{g_1 + g_2}= z^{g_1} z^{g_2} $
		\end{nota}
	\end{frame}
	
	\section{La costruzione di Greither}
	
	\begin{frame}[fragile]{La funzione $\beta$}
		Dato $ n= p_1 ^{e_1} \cdots p_s ^{e_s} $ definiamo: 
		\begin{itemize}
			\item $ S = \{1, ... , n \}$
			\item $ \PS = \{ I \,|\, I \subsetneq S\}$ 
			\item $ n_I = \prod_{i \in I} p_i ^{e_i} $ 
		\end{itemize} \pause
		Consideriamo una funzione $$ \beta : \PS \to \Z[G_0] $$ che utilizzeremo per costruire il sottogruppo cercato. \pause 
		\begin{deff}
			$\beta$ si dice moltiplicativa se $ \beta (\emptyset) = 1 $ e dati $ I,J $ con intersezione vuota abbiamo $ \beta (I\cup J) = \beta(I) \beta(J)$.
		\end{deff}
	\end{frame}

	\begin{frame}{Le unità di Greither}
		Definiamo ora, per ogni $ a \in (1 , n/2)$ coprimo con $ n $ l'unità reale:
			\begin{exampleblock}{}
			\[
				\xi_a (\beta) := \zeta ^{d_a (\beta)} \frac{\sigma_a (z(\beta))}{z(\beta)} \text{ con } d_a(\beta)= (1-a)\frac{t}{2}
			\]
			\end{exampleblock}
			dove abbiamo che
			\begin{itemize}
				\item $ z_I  := 1 - \zeta ^{n_I}$
				\item $ z(\beta ):= \prod_{i\in I} z_I ^{\beta(I)} $
				\item $ \sigma_a(\zeta)= \zeta ^a $
			\end{itemize}
	\end{frame}	
	
	\begin{frame}{Il gruppo di Greither}
		\begin{alertblock}{}
		 		$C_\beta $ è il gruppo generato da:
		 		\begin{center}
		 		
		 		 $ {-1} $ e $ \xi_a (\beta) $ per $ 1< a< n/2 $ e $ (a,n)=1 $
		 		\end{center}
	 	\end{alertblock}
	 	\pause
	 	Per l'indice useremo la notazione $$ [E_K : C_\beta] = h_K i_\beta $$
	\end{frame}
	
	\begin{frame}{Buona definizione di $C_ \beta $}
		Possiamo limitarci a definire $\beta$ su $ \Z[G] $ e poi considerare un sollevamento (in inglese lift)\\
		\pause
		\begin{lem}
			Se due funzioni $ \beta_1 $ e $ \beta_2 $ coincidono su $ \Q(\zeta_{n/n_I} )$\footnote{$ \zeta_{n/n_I} = \zeta_n^{n_I} $} per ogni $ I \in \PS $ allora le unità $ \xi_a (\beta) $ sono uniche a meno del segno
		\end{lem}
	\end{frame}
	
	\subsection{Calcolo dell'indice}
	
	\begin{frame}{Caso generale}
		\begin{teo}
				Data una funzione $ \beta : \PS \to \Z [G] $ segue che
					\begin{equation}
					\label{eq:idx1}
						i_\beta = \prod_{ \substack{\chi \neq 1 \\ \text{pari}}} \left( \sum_{\substack{ I \in \PS \\ (f_\chi , n_I) = 1}} \phi (n_I) \cdot \chi (\beta (I)) \cdot \prod_{i \not \in  I} (1- \chi^{-1} (p_i)) \right) 
					\end{equation}
		\end{teo}\pause
		Dove abbiamo che:
		\begin{itemize}
			\item $\phi$ è la funzione di Eulero 
			\item $\chi$ si dice \textit{pari} se $ \chi (-1) = 1 $
			\item Con $ \chi^{-1} $ si intende il carattere che vale $ 1/\chi $ sugli elementi invertibili
		\end{itemize}
	\end{frame}	
	
	
	
	
	
	
	
	
	
	
\end{document}
