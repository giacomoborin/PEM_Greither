\documentclass{beamer}
\usepackage[T1]{fontenc}
\usepackage[utf8]{inputenc}
\usepackage[italian]{babel}

\usepackage[style=numeric, maxnames=4,backend=bibtex]{biblatex}
% other styles: numeric authortitle
\addbibresource{../biblio.bib}

\title{Greither unit undex}
\author{Giacomo Borin}
%\subtitle{}
%\date{...}
\institute{Università di Trento}
%\titlegraphic{...} 

\usetheme{Madrid}
\usecolortheme{rose}
\setbeamercovered{dynamic}

\AtBeginSection{
	\begin{frame} 
		\sectionpage
	\end{frame}
}


% Commands 
\newcommand{\PS}{\mathcal{P}_S}
\newcommand{\Z}{\mathbb{Z}}
\newcommand{\F}{\mathbb{F}}
\newcommand{\K}{\mathbb{K}}
\newcommand{\ZZ}[1]{\mathbb{Z}_{#1}}
\newcommand{\Q}{\mathbb{Q}}
\newcommand{\C}{\mathbb{C}}
\newcommand{\R}{\mathbb{R}}

\DeclareMathOperator*{\eqb }{=}
\DeclareMathOperator{\Gal}{Gal}
\DeclareMathOperator{\ord}{ord}


\begin{document}

	\begin{frame}[plain]
	    \maketitle
	\end{frame}
	
	\begin{frame}{Introduzione}
		In questo lavoro ho rielaborato l'articolo: 
		\pause
%		\printbibliograph
		\begin{exampleblock}{}
			\fullcite{GRE}
		\end{exampleblock}	
		\pause
		Completando i prerequisiti richiesti per la comprensione e implementando alcuni calcoli in \href{https://www.sagemath.org}{\includegraphics[height= 11pt]{../images/sage.png} }
		
	\end{frame}
	\section{Prereqisiti}

	\begin{frame}
		\begin{block}{Primo oggetto di interesse: $ E_K $}
			Il gruppo delle unità di $ K $, il sottocampo reale massimo di $ \Q (\zeta_n )  $
		\end{block}
	\end{frame}

\end{document}
