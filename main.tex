\documentclass[]{article}
\usepackage[T1]{fontenc}
\usepackage[utf8]{inputenc}
\usepackage[english]{babel}
\usepackage{amssymb,amsmath,amsthm,mathtools} 
\usepackage{tikz-cd,wrapfig}
\usepackage{tcolorbox}

%%Color
%\usepackage{xcolor}
%\pagecolor[rgb]{0,0,0} %black
%\color[rgb]{1,1,1} %grey



% Theorem definitons 
\theoremstyle{plain}
\newtheorem{teo}{Theorem}[section]
\newtheorem{lem}[teo]{Lemma}
\newtheorem{prop}[teo]{Proposition}
\newtheorem{cor}[teo]{Corollary}
\newtheorem*{form}{Formula}

\theoremstyle{remark}
\newtheorem{rem}{Remark}
\newtheorem{rems}[rem]{Remarks}

\theoremstyle{definition}
\newtheorem{deff}[teo]{Definiton}
\newtheorem{idea}{Idea}
\newtheorem*{nota}{Notation}


\usepackage[style=numeric, maxnames=4,backend=bibtex]{biblatex}
% other styles: numeric authortitle
\addbibresource{biblio.bib}
\usepackage{hyperref}

% Commands 
\newcommand{\PS}{\mathcal{P}_S}
\newcommand{\Z}{\mathbb{Z}}
\newcommand{\F}{\mathbb{F}}
\newcommand{\K}{\mathbb{K}}
\newcommand{\ZZ}[1]{\mathbb{Z}_{#1}}
\newcommand{\Q}{\mathbb{Q}}
\newcommand{\C}{\mathbb{C}}
\newcommand{\R}{\mathbb{R}}

\DeclareMathOperator*{\eqb }{=}
\DeclareMathOperator{\Gal}{Gal}
\DeclareMathOperator{\ord}{ord}


%opening
\title{The Greither unit index,\\
an undergraduate overview and \\
a \textit{sagemath} implementation}
\author{Giacomo Borin}

  
\begin{document}

\maketitle

\begin{abstract}
 	\nocite{GRE}
	In this work I show a rielaboration of the article: \\
	\begin{center}
	Cornelius Greither, \textit{Improving Ramachandra's \\ 
	and Levesque's unit index}, 1999\\
	\end{center}
	I've introduced the principal instruments and results from Galois Theory and Number Theory necessary for the comprension 
	
	
\end{abstract}

\tableofcontents
\newpage

\section{Introduction to the working set}
	\subsection*{}
	
	Consider the $ n $-th cyclotomic field $ \Q (\zeta_n) $ with $\zeta_n$ a $ n $-th primitive root of unity, with $ n \not \equiv 2 \mod 4 $, and define $ K $ as the maximal real subfield of $ \Q (\zeta) $, also another notation that we will use for the maximal real subfield is $ \Q(\zeta_n)^+$. From now we will refer to $ \zeta_n  $ without the index if not necessary.
	
	\begin{prop}
		The maximal real subfield is $ K = \Q (\zeta + \zeta ^{-1}) $
	\end{prop}
	
	
	\begin{wrapfigure}{l}{0.5\textwidth}
		\label{fig:ramification}
		\begin{tikzcd}[column sep=small]
		& {\Q (\zeta)}                            &                               \\
		{K} \arrow[ru, "2"] \arrow[r] & {H} \arrow[u, "2" ] & {K'} \arrow[lu, "2"'] \arrow[l] \\
		& {\Q} \arrow[lu] \arrow[ru]      &                              
		\end{tikzcd}
	\end{wrapfigure}
	
	
	\begin{proof}
		First of all we can easly see that $ K $ is real, infact since for the root of unity $ \overline{\zeta} = \zeta ^{-1} $ (complex conjugation) and so:
		\begin{equation*}
		\overline{ \zeta + \zeta^{-1} }= \overline{ \zeta} + \overline{\zeta^{-1} } = \zeta ^{-1}  + \zeta
		\end{equation*}
		So $ \zeta + \zeta ^{-1} $ is real and $ K $ too.\\
		Since $ \Q (\zeta) $ is not real we have that the index $ e := [\Q (\zeta)  : K] \geq 2 $ (strictly greater than $ 1 $). \\
		Consider now the polynomial of degree 2 in $ K[x] $ : $ f = (x-\zeta)(x- \zeta^{-1}) = x^2 - (\zeta + \zeta ^{-1})x +1  $, since $\zeta$ is a root obviosly $ e \leq 2 $, so the subfield $ K $ has maximal degre since this is the minimal degree for a proper subfield. \\
		If there was another $ K' = \Q ( \chi ) $ with such property we can consider $ H = \Q (\zeta, \chi) $ that is also real with $ \Q (\zeta) \supsetneq H \supset K $, so $ H=K $ and akin $ H = K' $ so $ K = K' $ and $ K $ is unique.
	\end{proof}
	
	Now we will consider the \textbf{group of units} $ E_K$ that is the group formed by the invertible elements of its ring of integers $ O_K^\ast $. Is it possible to characterize the ring of integers for K \cite[Proposition~2.16]{CF} similiarly to what happens for $ O_{\Q (\zeta)} $ (infact the proof follows without difficulty from this)
	
	
	\begin{prop}
		$ O_K = \Z[\zeta + \zeta ^{-1}] $
	\end{prop}
	
	
	Since $ x^n - 1 $ is separable $ \Q (\zeta)  / \Q $ is a Galois extension and it's easy to see that its Galois group $ G_0  $ is isomorphic to $ ( \Z_{n} )^\ast $. Also we can see that:
	
	\begin{prop}
		$ K / \Q $ is a Galois extension and its Galois group $ G $ is isomorphic to $ \Z_{n}^*/ \{\pm 1\} $
	\end{prop}
	
	\begin{proof}
		Consider the map $ \sigma : G_0 \to G $ that maps $ \alpha_i $ to $ {\alpha_i }_{|_{K}} $ where $ \alpha_i $ is the automorphism that maps $ \zeta $ to $ \zeta ^i $. Obviously $\sigma$ is a morphism of groups.
		Also it is easy to describe its kernerl:
		\begin{align*}
			\ker (\sigma) =	& \{ \alpha_i \in G_0 \,|\,  x = \alpha_i(x)  \,\text{ for all } x \in K  \}\\
							 \eqb ^{(1)}	& \{ \alpha_i \in G_0 \,|\,  \zeta + \zeta^{-1} = \alpha_i (\zeta + \zeta^{-1}) = \zeta^i  + \zeta ^{-i}\} \\
							 \eqb^{(2)}& \{ \alpha_1 , \alpha_{-1}\}
		\end{align*}
		Where $ (1) $ follows from the fact that $ K = \Q (\zeta + \zeta ^{-1}) $ and $ (2) $ from linear algebra. %TODO
		So from the first theorem of isomorphism $ \sigma(G_0) \simeq  \Z_{n}^*/ \{\pm 1\}  $ and then 
		\begin{equation*}
			\phi(n)/2 = | \Z_{n}^*/ \{\pm 1\}| \leq |G| \leq [K : \Q] = [\Q( \zeta) : \Q ]/ 2 = \phi(n)/2
		\end{equation*}
		So $  \sigma(G_0) = G $ and $ |G| = [K : \Q] $ and the thesis follows. 
	\end{proof}
	
	\begin{rem}
		We excluded the case of $ n \equiv 2 \mod 4 $ because it is a repetition, infact in this situation $ G_0 \simeq \Z_{2 + 4k}^* $ and since $ 2+ 4k = 2 (1+2k) $ with the second term odd for the Chinese reminder theorem $ \Z_{2 + 4k}^* \simeq \Z_2^* \times \Z_{1 + 2k}^* \simeq \{1\} \times \Z_{1 + 2k}^* \simeq \Z_{1 + 2k}^* $ that is isomorphic to the Galois group for the $ n/2 $-th root of unity.  
	\end{rem}



	\subsection{The circluar units and the class number}

	\begin{deff}
		If $ \K $ is a number field (as $ \Q (\zeta) $ and $ K $)  we can define the \textbf{ideal class group} as the quotient $ \mathcal{F}_\K / \mathcal{P}_\K $ where:
	\begin{itemize}
		\item[$ \mathcal{F}_\K $] is the group of the nonzero fractional ideals of the ring of integers  $ O_\K $, that are the $ O_\K  $-submodules $ J $ of $ \K $ such that exists $ r \in O_\K  $ such that $ r I \subset O_\K  $
		\item[$ \mathcal{P}_\K $] is the set of nonzero principal fractionary ideals, so the ideals generated by only one element
	\end{itemize}
	\end{deff}


	We will indicate the number of classes in $ \mathcal{F}_\K / \mathcal{P}_\K $ as $ h_K $. This number will measure the \textit{"distance"} of $ O_\K $ to became a unique factorization domain. In \cite[Page~141]{RIN} it is proven that actually the ideal class group is finite so $ h_K $ is well defined. 

	
	\begin{deff}
		For a field $ \K \subseteq \Q (\zeta_n) $ (with n minimal) we define the group of cyclotomic (or circular) units as the intesection $ C_\K $ of the group generated by:
		\begin{equation*}
			\{ -1 ,  \zeta , \, 1 - \zeta ^a \text{ for } a = 1, ... , n-1 \}
		\end{equation*}
		and the unit group of $ O_\K $ ( $ E_{\K} $ ). An elements of $ C_\K $ is said to be a \textbf{circular unit} of $ \K $. 
	\end{deff}

	In general the circular units aren't easy to describe, infact in general $ 1 - \zeta ^a $ is not a unit, but for the particular case in which $ \K $ is the maximal real subfield ( $ K $ ) it has some intresting properties and it's related to the class number.
	
	
	If $ n=p^m $ where \textit{p} is a prime it is possible to describe (\cite[Lemma~8.1, Theorem~8.2]{CF}) explicitly the group of circluar units as the group generated by $ -1 $ and:
	\[
		\xi_a = \zeta^{ \frac{1-a}{2}} \frac{1 - \zeta ^a}{1 - \zeta } \text{ for } 1 < a < \frac{p^m}{2}, (a,p)=1
	\]
	Also we have the equality for the index:
	\begin{equation*}
		[ E_K : C_K ] = h_K
	\end{equation*}
	Moreover Sinnot in \cite{SIN} has generalized this to arbitrary $ n $ by showing that $ E_K / C_K $ is finite and the index is:
	 \begin{equation*}
		 [ E_K : C_K ] = 2^a h_K
	 \end{equation*}
	 where if $ g $ is the number of distinct primes dividing $ n $ we have that $ a=0 $ if $ g=1 $ (as expected) and $ a = 2^{g-2} + 1 - g $ otherwhise. 
	 Even if the index is simple does not exist a simple costruction of $ C_K $, so we have the problem:
	 
	 \begin{tcolorbox}
	 	Explicitly construct a group $ C' $ with finite index $ [ E_K : C' ] $ that is \textit{optimal}
	 \end{tcolorbox}
 
 	Where we will understand later what we mean by \textit{optimal}, but essentially we want the index to be small and with a simple factorization for $ [ E_K : C' ] / h_K $. In particular the costruction of Greither will generalize the work of Ramachandra and Levesque, so we will omit them from now and see them later. 
	 
	\subsection{Dirichlet Characters}
	 
	 \begin{deff}
	 	Given a group \textit{X} and a field $ \F $ a Dirichlet character is a group homomorphism $ \chi :X \to \F ^*  $
	 \end{deff}
	 
	 %TODO metti in ordine
	 In our case the field is $ \C $ and \textit{X} is the Galois group $ G_0 \simeq \Z_{n}^*$, so we can see the Dirichlet characters as homomorphisms: $ \xi :  \Z_{n}^* \to \C ^* $. Since if $ n | m $ there is a natural homomorphism $ \Z_m^* \to \Z_n^* $, composing it with $ \chi $ we can induct a new character $ \hat{\chi} :  \Z_m ^*  \to \C^\ast $. This characters are completely equivalent. So we can define \textit{k} to be minimal positive integer such that exists a character $ \chi' :  \Z_k ^*  \to \C^\ast $ equivalent to $\chi$, and call it the \textbf{conductor} of $\chi$, denoted by $ f_\chi := k $. 
	 
	 In some cases the character are also extended as ring homomorphisms from $ \Z_n \to \C $, assuming $ \chi $ to be zero on the non invertible elements. In this way the conductor can be seen as a sort of period, infact for all $ n $ we have $ \chi(n)=\chi(n+f_\chi) $. 
	 
	 Also we need another object: the group ring $ \Z[G] $, that is a free $ \Z $-module with $ G $ as basis on which we define the addition (using the module addition)  and the moltiplication inducting it from the operation of \textit{G}.
	 This costruction is also possible for a general ring and a multiplicative group:
	 
	 \begin{deff}
	 	The group ring of \textit{X} over \textit{R}, denoted by $ R[X] $ or $ RX $, is the set of all mapping $ f : X \to R $ with finite support (i.e. with finite $ x \in X $ such that $ f(x) \neq 0 $). The addition and the scalar multiplication are defined as usual. 
	 \end{deff}
 
 	We can also have a group structure over $ R[X] $ using the vector addition and the multiplication: were $ fg $ is defined as: $ fg(x) = \sum_{ y \in X } f(y)g(y^{-1}x) = \sum_{uv = x} f(u)g(v)$.\\
 	This is only a formal representation of the linear combinations, useful for the definition, but we will obviosly use a simpler notation $ f = \sum_{x \in X} f(x) x $.
 	
 	Now we would like to generalize again the characters as ring homorphism from $ \Z[G] $ (or another Galois group) to $ \C $. This is very simple since \textit{G} is a basis for the free $ \Z $-module its definition over the group is enougth. 
	
	\begin{nota}
		Given the elements $ z \in \Q (\zeta) $ and $ f \in \Z[G_0]  $ it's well defined the power notation $ z^f $, infact for $ g\in G_0 $ we have the well definiton for $ z^g = g(z) $ , $ z ^{g_1 + g_2}= z^{g_1} z^{g_2} $ and $ z^{-g} = (z^g)^{-1} $. 
	\end{nota}
	
%	\subsection{Bho}
%	\begin{deff}
%		Let $ G $ be a group and $ R $ a commutative ring, let's consider the \textit{augmentation map} $\epsilon : R[G] \to R$ that sends every $ g \in G $ to $ 1_R $ and every $ r \in R $ to itself and its an homomorphism of $ R $-modules. We also say that the kernel of $\epsilon$ is the \textit{augmentation ideal}
%	\end{deff}

\section{The Greither Setup}

	\subsection*{}
	Let's consider an integer n (with $n \not \equiv 2 \mod 4$), with factorization $ n= p_1 ^{e_1} \cdots p_s ^{e_s} $ and let $ S = \{1, ... , s \}$. We will use the power set $ \PS = \{ I \,|\, I \subsetneq S\}$ and the notation $ n_I = \prod_{i \in I} p_i ^{e_i} $ 
	
	The Greither's idea is to define a subgroup starting from a function $ \beta : \PS \to \Z[G_0] $, then varing $\beta$ we have different subgroups but with similiar properties. 
	
	\begin{deff}
		A function $\beta$ is called multiplicative if $ \beta (\emptyset) = 1 $ and for all sets $ I,J $ with empty intersection we have $ \beta (I\cup J) = \beta(I) \beta(J)$.
	\end{deff}

	A multiplicative function is uniquely determinated by its values on the singletons: $ \{\{i\} \,|,\, i \in S\} $ (we will use this later for a particular construction)

	Consider a general function $\beta$ and $ I \in \PS $, we define $ z_I  := 1 - \zeta ^{n_I}$ and 
	$$ z(\beta ):= \prod_{I \in \PS } z_I ^{\beta(I)} $$ 
	Using that $ 1 - \zeta ^{-m} = -\zeta^{-m} ( 1 - \zeta ^m ) $ , $ \overline{ \zeta} = \zeta ^{-1}  $ and the properties of complex conjugation we have that 
	\begin{multline}\label{eq:zbetacon}
		\overline{z(\beta )} = \prod_{I \in \PS} ({1 - \zeta ^{-n_I}})^{\beta(I)}  = \prod_{I \in \PS} - \zeta ^{-n_I \beta(I)} (1 - \zeta ^{n_I})^{\beta (I)} =\\
		=(-1)^{|\PS|}\prod_{I \in \PS} \zeta ^{-n_I \beta(I)} z_I^{\beta (I)} \eqb^\ast - \zeta^{-t} z(\beta) \text{ with } t = \sum n_I \beta(I)
	\end{multline}
	In $\ast$ we use that $ |\PS| = 2^s -1 $ is odd. 
	
	We define now for $ a \in (1 , n/2)$ coprime with $ n $ the real unit:

	\begin{equation}\label{eq:xi}
		\xi_a (\beta) := \zeta ^{d_a (\beta)} \frac{\sigma_a (z(\beta))}{z(\beta)} \text{ with } d_a(\beta)= (1-a)\frac{t}{2}
	\end{equation}
	Where $\sigma_a$ is the automorphism $ \zeta \mapsto \zeta ^a $. This is real because using the equation \ref{eq:zbetacon} and $ \overline{\sigma_a(z)}= \sigma_a(\overline{z}) $ we have:
	\begin{equation}
		\overline{\xi_a(\beta)} = \zeta ^{-d_a(\beta)} \frac{\zeta^{-at} \sigma_a (z(\beta))}{\zeta^{-t} z(\beta)} = \xi_a(\beta)
	\end{equation}
 	And its a unit because its the product of circular units. %TODO
 	We now use this units to define the goal group of the article:
 	\begin{tcolorbox}
 		$C_\beta $ is the group generated by \textit{-1} and $ \xi_a (\beta) $ for $ 1< a< n/2 $ and $ (a,n)=1 $
 	\end{tcolorbox}
 	For its index we will use the notation: $ [E_K : C_\beta] = h_K i_\beta $. 

	\subsection{A little remark}
	
	Sometimes it is easier to work with functions $\beta$ to $ \Z [G] $ instead of $ \Z [G_0] $ and than consider a lift $ \overline{\beta} : \PS \to \Z [G_0] $. Obviously this is not unique, but this is not a problem because we can show that $ C_\beta $ remain the same. 
	
	\begin{rem}
		Given two set $ Z, \, Y $ and a function $ \phi : Y \to Z $ we say that $ f' : X \to Y $ is a \textbf{lift} of $ f : X \to Z $ if $ f = \phi \circ f' $, i.e. if the following diagram commute: 
		\[\begin{tikzcd}
		 & {Y} \arrow[d, " \phi"] \\
		{X} \arrow[r, "f"'] \arrow[ru, "f'"] & {Z}               
		\end{tikzcd}\]
		In our case $\phi$ is the morphism inducted on the group ring by the projection $G_0 \simeq  \Z_n^\ast  \to  \Z_n^\ast / \pm 1  \simeq G$
	\end{rem}
	
	Initally we can osserve that we can factor the real unit $\xi_a(\beta)$ with simpler real units
	\begin{equation*}
		x_a(\beta , I) = \zeta ^{\frac{(1-a)}{2} n_I \beta (I)} \frac{\sigma_a (	z_I ^{\beta(I)})}{z_I ^{\beta(I)}} 
	\end{equation*}
	such that we have the equality:
	\begin{equation}\label{eq:fact_xi}
		 \xi_a(\beta) = \prod_{I \in \PS}  x_a(\beta , I)
	\end{equation} 
	
	\begin{lem} \label{lem:gooddef}
		Consider two functions $ \beta_1 $ and $ \beta_2 $ from $ \PS $ to $ \Z[G_0] $ such that for all $ I \in \PS $ their images of $ \beta_i (I)$ coincides in $ \Z[\Gal( \Q(\zeta_{n/n_I})^+ / \Q )] $ \footnote{Observe that $ \Q(\zeta_{n/n_I})^+ $ is a subfield of $ K $ since $ \zeta_{n/n_I} =  \zeta_n^{n_I}$, and since we see the elements of the group rings as homomorphism of fields make sense to compare two elements for their image on $\Q(\zeta_{n/n_I})^+ $ %TODO make sense??
		} for $ i= 1,2 $. Then for all $ I \in \PS $ $ x_a(\beta_i, I) $ coincides for $ i=1,2 $
	\end{lem}

	\begin{proof}
		Obviously for all $ I \in \PS $ $ 	x_a(\beta_i , I) $ depends only on the image of $ \beta_i $ over $ z_I = 1- \zeta_n ^{n_I} \in \Q(\zeta_{n/n_I}) $ \footnote{ $ \zeta_{n/n_I} =  \zeta_n^{n_I} $}, so it's enougth to show the equivalence over $  \Q(\zeta_{n/n_I}). $ Since the two functions are equal on $ \Q(\zeta_{n/n_I})^+ $ their difference $ \beta_1(I) - \beta_2 (I)  $ is the identity on the reals, so it is a multimple of $ 1 - j $, where $ j $ is the complex conjugation %TODO Non ho ben capito perchè 
		We can observe now, using morphism properties, that exist a unit $ r $ such that:
		\begin{equation*}
			\Q (\zeta_{n/n_I})^+ \ni q = \frac{x_a(\beta_1 , I)}{x_a(\beta_2 , I)} = {\left( \zeta ^{\frac{(1-a)}{2} n_I } \frac{\sigma_a (	z_I )}{z_I } \right)  }^{\beta_1(I) - \beta_2 (I)} = r ^{1-j}
		\end{equation*}	
		So we have that $ \overline{q}=q^j = r^{(1-j)j} = r^{j  - 1 }= q^{-1}$ (since $j^2 = 1 $), that for real numebers happen only for $ \pm 1 $
	\end{proof}

	\begin{rem}
		For what we have seen in the equation \ref{eq:fact_xi} it follows immediatly that also $ \xi_a(\beta) $ is unique up to a sign if $\beta$ is a lifting of a function from $ \PS $ to $ \Z[G] $. Since the group $ C_\beta $ contains $ -1 $ it is enougth to have a function $ \beta : \PS \to \Z[G]$ for its definition.
	\end{rem}

	\section{Index calculation}
	\subsection*{}
	\begin{teo}
		\label{teo:idx1}
		For any function $ \beta : \PS \to \Z [G] $ we have
		\begin{equation}
		\label{eq:idx1}
			i_\beta = \prod_{ \substack{\chi \neq 1 \\ \text{even}}} \left( \sum_{\substack{ I \in \PS \\ (f_\chi , n_I) = 1}} \phi (n_I) \cdot \chi (\beta (I)) \cdot \prod_{i \not \in  I} (1- \chi^{-1} (p_i)) \right) 
		\end{equation}
	\end{teo}

	\begin{rems}[On theorem \ref{teo:idx1}]
		\begin{itemize}
			\item $\phi$ is the Euler totient function
			\item A character $\chi$ is said to be \textbf{even} if $ \chi (-1) = 1 $
			\item With $ \chi^{-1} $ we mean the character defined as $ 1/\chi $ on the invertible elements and zero otherwhise, that is also a morphism because $ 1/(xy) = (1/x)(1/y)$.
			
			
		\end{itemize}
	\end{rems}

	For the proof we need the following Lemmas:
	
	\begin{lem} \label{lem:fact}
		For $ z \in \Q (\zeta )^\ast  $ and $ \gamma \in \Z [G_0] $, then for any character $\chi$ we have:
		\begin{equation} \label{eq:fact}
			\sum_{(a,n)=1} \chi ^{-1} (a) \log | z ^{\sigma_a \gamma }| = \chi (\gamma) \sum_{(a,n)=1} \chi ^{-1}(a) \log | z ^{\sigma_a  }|
		\end{equation}
	\end{lem}
	\begin{proof}
		It is easy to prove this for $\gamma = \sigma_g \in G_0$, infact since $ g $ is invertible in $ \Z_{n} $ is possible to change the index from $ (a,n)=1 $ to $ (ag,n)=1 $ and rearrange. Then we can pass to $ \Z[G_0] $ using the additivity of $ \chi $ and the logaritm of exponential (also the modulo is multiplicative).
	\end{proof}

	For the calculation of the index we need a new object that allows to evaluate a :
	\begin{deff}
		The \textbf{regulator} $ R_L $ of a number fields $ L $ is defined as follows:
		given its rank $ r $, a set of independent units $ \{\epsilon_1 , ... , \epsilon_r\} \subset L$ and $ \{ \sigma_1 , ... , \sigma_{r+1} \} $ its embedding into $ \R $ or $ \C $. Set $ \delta_j $ to be $ 1 $ if $ \sigma_j $ is real, and $ 2 $ otherwhise. \newline
		Then:
		\begin{equation}
			R_L(\epsilon_1 , ... , \epsilon_r) = | \det (\delta_i \log | \epsilon_j ^{\sigma_i}|)_{1\leq i,j\leq r} |
		\end{equation}
	\end{deff}
	
	\begin{rem} \label{rem:reg}
		The embedding that we decide to omit is not relevant, infact since they are units their norm is 1, so $ \sum_i  \delta_i \log | \epsilon_j ^{\sigma_i}| =\log | \prod_i \epsilon_j^{\delta_i \sigma_i }|=\log |N(\epsilon_j)| = 0$, so writing this equality as a linear system from Cramer formula follows the uniqueness of the determinant up to a sign. %TODO why the \delta_I ???
	\end{rem}

	Now we need to recall some Lemmas from \cite{CF} without the proofs: 
	
	\begin{lem}[Lemma 4.15 in \cite{CF}]
		\label{lem:index_reg}
		Given the groups $ A \subset B $ of finite index, generated by independent units of a number field $ L $, respectively $ \{\epsilon_i\}_{i=1}^r $ and $ \{\mu_i\}_{i=1}^r $:
		\begin{equation} \label{eq:index_reg}
			[B:A]= \frac{R_L(\epsilon_1 , ... , \epsilon_r) }{R_L(\mu_1 , ... , \mu_r) }
		\end{equation}
	\end{lem}

	\begin{lem}[Lemma 5.26 in \cite{CF}] 
		\label{lem:det}
		Let \textit{X} be a finite abelian group and let $ f $ be a function on \textit{X} with values in $ \C $
		\begin{equation}\label{eq:det}
			\det (f (\sigma \tau ^{-1}) - f (\sigma))_{\sigma , \tau \neq 1 } = \prod_{\substack{\chi \in \hat{X} \\ \chi \neq 1}}  \sum_{\sigma \in X} \chi(\sigma)f(\sigma)
		\end{equation}
		Where $ \hat{X} $ is the set of homorphisms (characters) from $ X $ to $ \C ^\ast $
	\end{lem}
	In our case $ X $  will be $ G \equiv \Z_n / \pm 1 $, and so the elements of $ \hat{X} $ are the even characters of $ \Z_n $. 
	
	\begin{proof}[Proof of Theorem \ref{teo:idx1} ]
			Using Lemma \ref{lem:index_reg} we can evaluate $ [E_K : C_\beta ] $ with the quotient of the regulators. In the equation \ref{eq:index_reg} we can omit the unit $ -1 $ since it is contained in both the two groups (for what we have said in \ref{rem:reg} can only change a sign). 
			
			So we need to prove that $ R(\xi_a(\beta)) = \pm R_K h_K A $ with $ (a,n)=1 \, , \,1 < a <n/2 $ and $ A$ be the right part of the equation \ref{eq:idx1}. The $\pm$ is a more simple way to indicate that we don't matter the sign without inserting everything in a modulo. 
			
			From definition, using that $ \delta_i $ is always $ 1 $ since the units are all real and the embeddings can be seen as elements of the Galois Group \textit{G}:
			\begin{align*}
				R(\xi_a(\beta))  & = \pm \det [\log |  \xi_a(\beta)^{\tau}|]											\quad ((a,n)=1 \, , \,1 < a <n/2 \, ; \, \tau \in G) \\
										& \eqb^{(1)} \pm \det [ f(\tau \sigma ) - f(\tau ) ]_{\sigma , \tau \in G - 1} 		\quad \text{with } f(\sigma)= \log |\sigma z(\beta)|\\
										& \eqb \prod_{\substack{\chi \neq 1 \\ \text{even}}}  \frac{1}{2} \sum_{ ( a , n)= 1} \chi^{-1}(a ) \log |\sigma_a z(\beta)|\ \quad \text{ using Lemma \ref{lem:det} }\\
										& \eqb \prod_{\substack{\chi \neq 1 \\ \text{even}}}  \frac{1}{2} \sum_{ ( a , n)= 1} \chi^{-1}(a ) \sum_{I \in \PS} \log | ( 1 - \zeta ^{n_i a} )^{\beta (I)}|\ \\
										& \eqb \prod_{\substack{\chi \neq 1 \\ \text{even}}}  \frac{1}{2}\sum_{I \in \PS} \left( \sum_{ ( a , n)= 1} \chi^{-1}(a )  \log | ( 1 - \zeta ^{n_i a} )^{\beta (I)}|\ \right) \\
										& \eqb^{\ref{eq:fact}} \prod_{\substack{\chi \neq 1 \\ \text{even}}}  \frac{1}{2}\sum_{I \in \PS} \left( \chi(\beta (I))\sum_{ ( a , n)= 1} \chi^{-1}(a )  \log | ( 1 - \zeta ^{n_i a} )|\ \right) 
			\end{align*}
			Where in $ (1) $ we have used that $ \log | \zeta ^ d| =0 $ because $\zeta$ is a unit and the logaritm's properties. 
			
			The last part is a bit technical and uses \cite[Lemma~8.4]{CF} to reduce the first sum to the $ I \in \PS $ such that $ (f_\chi , n_I) = 1$, and then continues as for the proof of Theorem 8.3 in \cite[Pages~148-150]{CF} and involves the analytic class numebr formula %TODO ask to the professor for more
			and Dirichlet L-series (also Chapter 4 in \cite{CF}). 
			\end{proof}
		
	\subsection{Particular case of formula \ref{eq:idx1}}

	Now we can try to see what happen if we request some conditions over $\beta$, with some particular cases.
	
	\begin{teo}
		\label{teo:idx_m}
		If we assume $ \beta : \PS \to \Z[G] $ to be multiplicative then:
		\begin{equation}\label{eq:idx_m}
			i_\beta =  \prod_{ \substack{\chi \neq 1 \\ \text{even}}} \left( \prod_{p_i \nmid f_\chi} \left( \phi (p_i^{e_i}) \cdot \chi (\beta (i))  + 1- \chi^{-1} (p_i)\right)  \right) 
		\end{equation}
		Where $ \beta(i) $ mean $ \beta (\{i\})$
	\end{teo}

	\begin{proof}
		It is easy that we can lift $ \beta $ to $ \Z [G_0] $ conserving multiplicativity. Consider now, for $ \chi \neq 1 $ even, the two factors :
		\begin{equation}\label{eq:Tchi}
			T_\chi = \sum_{\substack{ I \in \PS \\ (f_\chi , n_I) = 1}} \phi (n_I) \cdot \chi (\beta (I)) \cdot \prod_{i \not \in  I} (1- \chi^{-1} (p_i)) 
		\end{equation}
		and 
		\begin{equation}\label{eq:Uchi}
			U_\chi = \prod_{p_i \nmid f_\chi} \left( \phi (p_i^{e_i}) \cdot \chi (\beta (i))  + 1- \chi^{-1} (p_i))\right)
		\end{equation}
		that are the arguments of the products in equations \ref{eq:idx1} and \ref{eq:idx_m}. So it's enough to prove $ U_\chi = T_\chi $. 
		Initially we can observe that the argument of the sum in \ref{eq:Tchi} are the subset of $ S_\chi = \{ i \,|\, p_i \nmid f_\chi \} $. Also we can observe 
		 \begin{align*}
		 	\phi (n_I ) & = \prod_{ i \in I} \phi(p_i^{e_i})\\
		 	\chi( \beta (I)) & = \chi \left(  \prod_{ i \in I} \beta(i) \right)  = \prod_{ i \in I} \chi(\beta (i))
		 \end{align*}
		 From which expanding the product of $ U_\chi $ we get the equality.
	\end{proof}

	Using this formula and the definition of $ C_\beta $ we can see that for $\beta \equiv 1 $ (that is the simplest example of multiplicative $\beta$) we get the Ramachandra's unit index from \cite{RAM} (or in a more modern notation \cite[Theorem~8.3]{CF} ):
	\begin{equation}\label{eq:idx_ram}
		[E_K : C_R ] = h_K \cdot  \prod_{ \substack{\chi \neq 1 \\ \text{even}}} \left( \prod_{p_i \nmid f_\chi} \left( \phi (p_i^{e_i})  + 1- \chi (p_i))\right)  \right) 
	\end{equation}
	Where $ C_R $ is the group generated by $ -1 $ and the units of the form of \ref{eq:xi} with $ \beta(I) = 1  $:
	\[ \xi_a := \zeta ^{d_a } \prod _{I \in \PS} \frac{ 1 - \zeta^{an_I}}{ 1 - \zeta^{n_I}}  \text{ with } d_a =\frac{1}{2}(1-a) \sum_{I \in \PS} n_I\]
	
	We can also construct $\beta$ multiplicative such that:
	\[ \beta(i) = \begin{cases} 1 \text{ if exists } \chi \neq 1 \text{ even, with } \chi(p_i)=1 \\ 0 \text{ otherwhise } \end{cases}\]
	And we obtain the Levesque group $ C_\mathcal{D} $ defined in \cite[Page~331]{LEV}
	
	\subsection{A new system of units}
	
	Following the previous steps we know construct a new multiplicative map $\beta$ with a more optimal index. 
	\begin{nota}
		If $ x $ is an element of finite group $\Gamma$ we define:
		\[ N_x := 1 + x + ... + x^{\ord (x) - 1} \in \Z[\Gamma]\]
		This will be called \textit{trace} element of $ x $.
	\end{nota}

	Let now define $ G_i $ for $ i=1,...,s $ to be the Galois group $ \Gal ( \Q(\zeta_{n/p_i^{e_i}})^+ / \Q)) $. Consider now the Frobenius automorphism $F_i \in G_i$:  
	
	\begin{alignat*}{2}
		F_i : \Q(\zeta_{n/p_i^{e_i}})^+ &\longrightarrow \: \Q(\zeta_{n/p_i^{e_i}})^+  \\
		\zeta_{n/p_i^{e_i}}  &\longmapsto \: \zeta_{n/p_i^{e_i}} ^ {p_i}
		\end{alignat*} 
	and its trace element $ N_{F_i} \in \Z [G_i]$. Now we choose for every $ i=1, ... , s $ a lift of $ N_{F_i} $ into $ \Z[G_0] $\footnote{Remind that $ \zeta_{n/p_i^{e_i}} = \zeta_n^{p_i^{e_i}}$} and associate it to $\beta(i)$; then $\beta$ is defined multiplicatively. 
	
	Of course $\beta$ is not unique, but for all $ I \in \PS $ they coincide in $ \Z[\Gal( \Q(\zeta_{n/n_I})^+ / \Q   )]  $, so we can use Lemma \ref{lem:gooddef} and $ C_\beta $ is well defined. 
	
	\subsection{A factorization for $ i_\beta $}
	
	Here we will recall some facts and definitions from \cite[Chapter~11]{RIN}. These are genaralities for a finite separable extension of a number field, but we will restrict in the case of $ \K / \Q$ number field. 
	
	Consider a prime $ p $ in $ \Z $ and its ideal extension in the ring of integers $ pO_\K $. Since $ O_K $ is a Dedekin domain we can factorize it with prime ideals:
	\begin{equation}\label{eq:fact_p}
		pO_\K = \prod_{j=1}^g \mathfrak{p}_j^{\epsilon_j}
	\end{equation} 
	
	\begin{deff} \label{def:degree}
		The number $ g $ is said to be the \textit{decomposition degree} (or number) of $ p $ in the extension $ \K / \Q $.
						
		For every $ j=1, ..., g $, $ \epsilon_j $ is said to be the \textit{ramification degree} (or index) of $ \mathfrak{p}_j $ in $ \K / \Q $.
						
		For every $ j=1, ..., g $, $ f_j := [O_\K / \mathfrak{p}_j : \Z_p ] $ is called the \textit{inertial degree} (or residual). 
	\end{deff}
	
	In particular is possible to prove that if $ n= [\K : \Q] $ so \[n = \sum_{j=1}^g f_j \epsilon_j \] Also if $ \K / \Q $ is a Galois extension then also $ \epsilon_j, f_j$ does not depend on $ j $ and so \[ n = \epsilon fg \]
	
	We also recall from \cite[Theorem~3.7]{CF} the relation between the characters $ X $ over the galois group of $ \K/\Q $ and decomposition degree of $ p $ in the extension $ \K / \Q $ :
	\begin{equation}\label{eq:g_char}
		g = | \{ \chi \in X \,|\, \chi(p)=1 \}
	\end{equation}
	
	
	We have now the ingredients for evaluating in a optimal way $ i_\beta $. 
	
	For $ i\in {1 , ... , s} $ define $ g_i , f_i , \epsilon_i $ to be as in the definition \ref{def:degree} for the prime $ p_i$ in $ K /\Q $.
	
	Is possible to show (fact A in \cite[Page~544]{RIN}) that the inertia degree $ f_i $ is closely realted with the Frobenius morphism, in fact
	\[ f_i = \ord (F_i)\]
	
	\begin{tcolorbox}
		\begin{teo} \label{teo:idx_opt}
			With $ C_\beta $ defined as before we have 
			\[ i_\beta  =  \prod_{i=1}^s \epsilon _i^{g_i - 1} f_i ^{2 g_i - 1}\]
		\end{teo}
	\end{tcolorbox}
	
	\begin{rem}
		This index is \textit{optimal} because we have a lot of info about its factorization for definiton, also since $ \epsilon _i $ and $ f_i $ are factors of $ n $, the factorization of the last one is enough to know the $ i_\beta $'s. We will see later that is also smaller than other index already studied. 
	\end{rem}
	
	\begin{proof}
		For $ s=1 $ this is trivial, since $ i_\beta = 1 $. 
		
		For $ s \geq 2 $ is possible to prove that $ \epsilon _i = \phi ( p_i ^ {e_i}) $. For $ i \in S$ and $ \chi $ such that $ p_i \nmid f_\chi $ we define
		 \[ y(\chi , i) = \phi(p_i^{e_i}) \cdot \chi (\beta (i))  + 1- \chi^{-1} (p_i) \]
		Considering $ \overline{\chi} $ to be the character induced by $ \chi $ in $ G_i$ we have $ \chi(\beta(i)) = \overline{\chi}(N_{F^i})  $ and $ \chi (p_i) = \overline{\chi}(F_i) $ using the isomorphism between $ G_i $ and the relative modulo ring. There are two cases: 
		\begin{itemize}
		\item[$ \chi(p_i) = 1 $ :] $ \overline{\chi}(N_{F_i}) = \sum \overline{\chi}(F_i)^j = \sum 1 = \ord (F_i) = f_i$  and so $ y(\chi , i) = \phi(p_i^{e_i})f_i + 0 = \epsilon _i f_i $
		\item[$ \chi(p_i) \neq 1 $ :] Since $ \overline{\chi}({F_i})  \neq 1 $ follows $ \overline{\chi}(N_{F_i}) = 0 $. %TODO Ma dove???
		Hence $ y(\chi , i) = \phi(p_i^{e_i}) \cdot \chi 0  + 1- \chi^{-1} (p_i)= 1- \chi^{-1} (p_i) $
		\end{itemize}
		Then, indexing the product by $ i $:
		\begin{align*}
			i_\beta & = \prod_{i=1}^s	\prod_{ \substack{\chi \neq 1 \text{ even}\\ p_i \nmid f_\chi }} y(\chi ,i)				\\	
			& = \prod_{i=1}^s		\left( \prod_{ \chi(p_i) = 1  } \epsilon _i f_i \prod_{\chi(p_i) \neq 1}  (1- \chi^{-1} (p_i))\right) 	\quad (\chi \neq 1 \text{ even, } p_i\nmid f_\chi )		\\	
			& \eqb ^\ast  \prod_{i=1}^s	((\epsilon _i f_i )^{g_i - 1} \cdot f_i ^{g_i})				\\	
			 & = \prod_{i=1}^s		\epsilon _i^{g_i - 1} f_i ^{2 g_i - 1}		
		\end{align*}
		In $\ast$ the exponent $ g_i - 1 $ come from \ref{eq:g_char} (there isn't the trivial character). Instead for the second part we are using that, since $ \chi^{-1}(p_i) = \overline{\chi^-1}({F_i})  $ is a non trivial $ \ord(F_i)=f_i $-th root of unity for all characters (and varing $ \chi $ we get every unit $ g_i $ times) we use that from the factorization of $ x^f - 1  $ as $ (x - 1 )(1 + x + ... + x^{f-1}) $ evaluated in $ 1 $ we have $ \prod_{\zeta^f=1, \zeta\neq 1} (1- \zeta) = f $.
		
		
	\end{proof}
	
	
	\newpage
	\printbibliography
\end{document}
